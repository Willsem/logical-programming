\section{ФОРМИРОВАНИЕ ОТВЕТА}

Для одного из вариантов ВОПРОСА и одного из заданий  составить таблицу, отражающую конкретный порядок работы системы: Т.к. резольвента хранится в виде стека, то состояние резольвенты требуется отображать в столбик: вершина – сверху! Новый шаг надо начинать с нового состояния резольвенты! Для каждого запуска алгоритма унификации, требуется указать № выбранного правила и дальнейшие действия – и почему.

{
\small
\begin{longtable}{|p{1.15cm}|p{6cm}|p{6cm}|p{4cm}|}
    \caption{moreThat([4, 7, 1, 2, 8], 5, ResultMore)} \\
    \hline
    № шага & Состояние резольвенты, и вывод: дальнейшие действия (почему?) & Для каких термов запускается алгоритм унификации: T1=T2 и каков результат (и подстановка) & дальнейшие действия: прямой ход или откат (почему и к чему приводит?) \\
    \hline
    1 & moreThat([4, 7, 1, 2, 8], 5, ResultMore) & Подстановка: Head = 4, Tail = [7, 1, 2, 8], Number = 5, Result = ResultMore & Прямой ход \\
      & & moreThat([4, 7, 1, 2, 8], 5, ResultMore) & \\
      & & moreThat([Head|Tail], Number, Result) & \\
    \hline
    2 & Head > Number & 4 > 5 & Обратный ход \\
      & moreThat(Tail, Number, TailResult) & & \\
      & Result = [Head|TailResult] & & \\
    \hline
    3 & moreThat([4, 7, 1, 2, 8], 5, ResultMore) & Подстановка: Head = 4, Tail = [7, 1, 2, 8], Number = 5, Result = ResultMore & Прямой ход \\
      & & moreThat([4, 7, 1, 2, 8], 5, ResultMore) & \\
      & & moreThat([Head|Tail], Number, Result) & \\
    \hline
    4 & Head <= Number & 4 <= 5 & Прямой ход \\
      & moreThat(Tail, Number, Result) & & \\
    \hline
    5 & moreThat(Tail, Number, Result) & Подстановка: Tail = [7, 1, 2, 8], Result = Result & Прямой ход \\
      & & moreThat([7, 1, 2, 8], 5, ResultMore) & \\
      & & moreThat([Head|Tail], Number, Result) & \\
    \hline
    6 & Head > Number & 7 > 5 & Прямой ход \\
      & moreThat(Tail, Number, TailResult) & & \\
      & Result = [Head|TailResult] & & \\
    \hline
    7 & moreThat(Tail, Number, TailResult) & Подстановка: Tail = [1, 2, 8], Number = 5, Result = TailResult & Прямой ход \\
      & Result = [Head|TailResult] & & \\
    \hline
    8 & Head > Number & 1 > 5 & Обратный ход \\
      & moreThat(Tail, Number, TailResult) & & \\
      & Result = [Head|TailResult] & & \\
      & Result = [Head|TailResult] & & \\
    \hline
    9 & moreThat(Tail, Number, TailResult) & Подстановка: Tail = [1, 2, 8], Number = 5, Result = TailResult & Прямой ход \\
      & Result = [Head|TailResult] & & \\
    \hline
    10 & Head <= Number & 1 <= 5 & Прямой ход \\
       & moreThat(Tail, Number, Result) & & \\
       & Result = [Head|TailResult] & & \\
    \hline
    11 & moreThat(Tail, Number, Result) & Подстановка: Tail = [2, 8], Number = 5, Result = Result & Прямой ход \\
       & Result = [Head|TailResult] & & \\
    \hline
    12 & Head > Number & 2 > 5 & Обратный ход \\
       & moreThat(Tail, Number, TailResult) & & \\
       & Result = [Head|TailResult] & & \\
       & Result = [Head|TailResult] & & \\
    \hline
    13 & moreThat(Tail, Number, Result) & Подстановка: Tail = [2, 8], Number = 5, Result = Result & Прямой ход \\
       & Result = [Head|TailResult] & & \\
    \hline
    14 & Head <= Number & 2 <= 5 & Прямой ход \\
       & moreThat(Tail, Number, Result) & & \\
       & Result = [Head|TailResult] & & \\
    \hline
    15 & moreThat(Tail, Number, Result) & Подстановка: Tail = [8], Number = 5, Result = Result & Прямой ход \\
       & Result = [Head|TailResult] & & \\
    \hline
    16 & Head > Number & 8 > 5 & Прямой ход \\
       & moreThat(Tail, Number, TailResult) & & \\
       & Result = [Head|TailResult] & & \\
       & Result = [Head|TailResult] & & \\
    \hline
    17 & moreThat(Tail, Number, TailResult) & Подстановка: Tail = [], Number = 5, Result = TailResult & Прямой ход \\
       & Result = [Head|TailResult] & & \\
       & Result = [Head|TailResult] & & \\
    \hline
    18 & ! & Отсечение & Прямой ход \\
       & Result = [Head|TailResult] & & \\
       & Result = [Head|TailResult] & & \\
    \hline
    19 & Result = [Head|TailResult] & Подстановка: Result = [8] & Прямой ход \\
       & Result = [Head|TailResult] & & \\
    \hline
    20 & Result = [Head|TailResult] & Подстановка: Result = [7, 8] & \\
    \hline
    21 & Пустая & \textbf{Результат:} MoreResult = [7, 8] & Обратный ход \\
    \hline
    22 & moreThat(Tail, Number, Result) & Подстановка: Tail = [8], Result = Result & Прямой ход \\
       & & moreThat([8], 5, ResultMore) & \\
       & & moreThat([Head|Tail], Number, Result) & \\
    \hline
    23 & Head <= Number & 8 <= 5 & Обратный ход \\
       & moreThat(Tail, Number, Result) & & \\
    \hline
    24 & moreThat(Tail, Number, Result) & Подстановка: Tail = [7, 1, 2, 8], Result = Result & Прямой ход \\
       & & moreThat([7, 1, 2, 8], 5, ResultMore) & \\
       & & moreThat([Head|Tail], Number, Result) & \\
    \hline
    25 & Head <= Number & 7 <= 5 & Обратный ход \\
       & moreThat(Tail, Number, Result) & & \\
    \hline
\end{longtable}
}

\section{ВЫВОДЫ}

Эффективность достигается за счет использования отсечения (!), которое останавливает поиски следующих фактов и правил. Так же за счет использования конструкции {\ttfamily [Head|Tail]} можно эффективно отделить голову списка от хвоста.
