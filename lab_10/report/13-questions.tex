\section{ВОПРОСЫ}

\begin{enumerate}
    \item \textbf{Как организуется хвостовая рекурсия в Prolog?}

        В языке Prolog рекурсия организуется при помощи правила, в котором есть обращение к тому же правилу.

    \item \textbf{Какое первое состояние резольвенты?}

        Первое состояние резольвенты -- вопрос.

    \item \textbf{Каким способом можно разделить список на части, какие, требования к частям?}

        В Prolog используется специальный символ для разделения списка на голову и хвост -- вертикальная черта |. Вертикальную черту можно использовать не только для отделения головы списка, но и для отделения произвольного числа начальных элементов списка.

    \item \textbf{Как выделить за один шаг первые два подряд идущих элемента списка? Как выделить 1-й и 3-й элемент за один шаг?}

        \begin{itemize}
            \item {[First, Second|\_]} (First -- первый элемент списка, Second -- второй)
            \item {[First, \_, Third|\_]} (First -- первый элемент списка, Third -- третий)
        \end{itemize}

    \item \textbf{Как формируется новое состояние резольвенты?}

        Преобразование резольвенты выполняется с помощью редукции.

        Редукция -- замена цели телом того правила, заголовок которого унифицируется с целью.
        Новая резольвента получается в два этапа:

        \begin{enumerate}
            \item В текущей резольвенте выберается одна из целей и для неё выполняется редукция $\Rightarrow$ получаем новую коньюнкцию целей(новую резольвенту)
            \item К полученной новой резольвенте применяется подстановка, как наибольший общий унификатор цели и заголовка правила, сопоставимого с этой целью.
        \end{enumerate}

    \item \textbf{Когда останавливается работа системы? Как это определяется на формальном уровне?}

        Работа системы останавливается в двух случаях:

        \begin{itemize}
            \item Когда встретился символ отсечения (!);
            \item Когда резольвента осталась пустой (формально не осталось подходящих фактов и правил).
        \end{itemize}
\end{enumerate}
