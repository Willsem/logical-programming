\section{ФОРМИРОВАНИЕ ОТВЕТА}

Для одного из вариантов ВОПРОСА и каждого задания  составить таблицу, отражающую конкретный порядок работы системы: Т.к. резольвента хранится в виде стека, то состояние резольвенты требуется отображать в столбик: вершина – сверху! Новый шаг надо начинать с нового состояния резольвенты!

{
\small
\begin{longtable}{|p{1.15cm}|p{4cm}|p{6cm}|p{6cm}|}
    \caption{factorial(4, Result)} \\
    \hline
    № шага & Состояние резольвенты, и вывод: дальнейшие действия (почему?) & Для каких термов запускается алгоритм унификации: T1=T2 и каков результат (и подстановка) & дальнейшие действия: прямой ход или откат (почему и к чему приводит?) \\
    \hline
    1 & factorial(4, Result) & Подстановка: N = 4, Result = Result & Прямой ход \\
    \hline
    2 & NextN = N - 1 & Подстановка NextN = 3 & Прямой ход \\
      & factorial(NextN, NextResult) & & \\
      & Result=N*NextResult & & \\
    \hline
    3 & factorial(NextN, NextResult) & Подстановка: N = 3, Result = NextResult & Прямой ход \\
      & Result=N*NextResult & & \\
    \hline
    4 & NextN = N - 1 & Подстановка NextN = 2 & Прямой ход \\
      & factorial(NextN, NextResult) & & \\
      & Result=N*NextResult & & \\
      & Result=N*NextResult & & \\
    \hline
    5 & factorial(NextN, NextResult) & Подстановка: N = 2, Result = NextResult & Прямой ход \\
      & Result=N*NextResult & & \\
      & Result=N*NextResult & & \\
    \hline
    6 & NextN = N - 1 & Подстановка NextN = 1 & Прямой ход \\
      & factorial(NextN, NextResult) & & \\
      & Result=N*NextResult & & \\
      & Result=N*NextResult & & \\
      & Result=N*NextResult & & \\
    \hline
    7 & factorial(NextN, NextResult) & Подстановка: N = 1, Result = NextResult & Прямой ход \\
      & Result=N*NextResult & & \\
      & Result=N*NextResult & & \\
      & Result=N*NextResult & & \\
    \hline
    8 & NextN = N - 1 & Подстановка NextN = 0 & Прямой ход \\
      & factorial(NextN, NextResult) & & \\
      & Result=N*NextResult & & \\
      & Result=N*NextResult & & \\
      & Result=N*NextResult & & \\
      & Result=N*NextResult & & \\
    \hline
    9 & factorial(NextN, NextResult) & Подстановка: N = 0, Result = NextResult & Прямой ход \\
      & Result=N*NextResult & & \\
      & Result=N*NextResult & & \\
      & Result=N*NextResult & & \\
      & Result=N*NextResult & & \\
    \hline
    10 & Result = 1 & Подстановка: Result = 1 & Прямой ход \\
       & Result=N*NextResult & & \\
       & Result=N*NextResult & & \\
       & Result=N*NextResult & & \\
       & Result=N*NextResult & & \\
    \hline
    11 & Result=N*NextResult & Подстановка: Result = 1 & Прямой ход \\
       & Result=N*NextResult & & \\
       & Result=N*NextResult & & \\
       & Result=N*NextResult & & \\
    \hline
    12 & Result=N*NextResult & Подстановка: Result = 2 & Прямой ход \\
       & Result=N*NextResult & & \\
       & Result=N*NextResult & & \\
    \hline
    13 & Result=N*NextResult & Подстановка: Result = 6 & Прямой ход \\
       & Result=N*NextResult & & \\
    \hline
    14 & Result=N*NextResult & Подстановка: Result = 24 & Прямой ход \\
    \hline
    15 & Пусто & \textbf{Результат:} Result = 24 & Обратный ход \\
    \hline
\end{longtable}
}

{
\small
\begin{longtable}{|p{1.15cm}|p{4cm}|p{6cm}|p{6cm}|}
    \caption{fibb(4, Result)} \\
    \hline
    № шага & Состояние резольвенты, и вывод: дальнейшие действия (почему?) & Для каких термов запускается алгоритм унификации: T1=T2 и каков результат (и подстановка) & дальнейшие действия: прямой ход или откат (почему и к чему приводит?) \\
    \hline
    1 & fibb(4, Result) & Подстановка: N = 4, Result = Result & Прямой ход \\
    \hline
    2 & PN = N - 1 & Подстановка: PN = 3 & Прямой ход \\
      & PPN = N - 2 & & \\
      & fibb(PN, PResult) & & \\
      & fibb(PPN, PPResult) & & \\
      & Result = PResult + PPResult & & \\
    \hline
    3 & PPN = N - 2 & Подстановка: PPN = 2 & Прямой ход \\
      & fibb(PN, PResult) & & \\
      & fibb(PPN, PPResult) & & \\
      & Result = PResult + PPResult & & \\
    \hline
    4 & fibb(PN, PResult) & Подстановка: N = 3, Result = PResult & Прямой ход \\
      & fibb(PPN, PPResult) & & \\
      & Result = PResult + PPResult & & \\
    \hline
    5 & PN = N - 1 & Подстановка: PN = 2 & Прямой ход \\
      & PPN = N - 2 & & \\
      & fibb(PN, PResult) & & \\
      & fibb(PPN, PPResult) & & \\
      & Result = PResult + PPResult & & \\
      & fibb(PPN, PPResult) & & \\
      & Result = PResult + PPResult & & \\
    \hline
    6 & PPN = N - 2 & PPN = 1 & Прямой ход \\
      & fibb(PN, PResult) & & \\
      & fibb(PPN, PPResult) & & \\
      & Result = PResult + PPResult & & \\
      & fibb(PPN, PPResult) & & \\
      & Result = PResult + PPResult & & \\
    \hline
    7 & fibb(PN, PResult) & Подстановка: N = 2, Result = PResult & Прямой ход \\
      & fibb(PPN, PPResult) & & \\
      & Result = PResult + PPResult & & \\
      & fibb(PPN, PPResult) & & \\
      & Result = PResult + PPResult & & \\
    \hline
    8 & Result = 1 & Подстановка: Result = 1 & Прямой ход \\
      & fibb(PPN, PPResult) & & \\
      & Result = PResult + PPResult & & \\
      & fibb(PPN, PPResult) & & \\
      & Result = PResult + PPResult & & \\
    \hline
    9 & fibb(PPN, PPResult) & Подстановка: N = 1, Result = PPResult & Прямой ход \\
      & Result = PResult + PPResult & & \\
      & fibb(PPN, PPResult) & & \\
      & Result = PResult + PPResult & & \\
    \hline
    10 & Result = 1 & Подстановка: Result = 1 & Прямой ход \\
       & Result = PResult + PPResult & & \\
       & fibb(PPN, PPResult) & & \\
       & Result = PResult + PPResult & & \\
    \hline
    11 & Result = PResult + PPResult & Подстановка: Result = 2 & Прямой ход \\
       & fibb(PPN, PPResult) & & \\
       & Result = PResult + PPResult & & \\
    \hline
    12 & fibb(PPN, PPResult) & Подстановка: N = 2 & Прямой ход \\
       & Result = PResult + PPResult & & \\
    \hline
    13 & Result = 1 & Подстановка: Result = 1 & Прямой ход \\
       & Result = PResult + PPResult & & \\
    \hline
    14 & Result = PResult + PPResult & Подстановка: Result = 3 & Прямой ход \\
    \hline
    15 & Пусто & \textbf{Результат:} Result = 3 & Обратный ход \\
    \hline
\end{longtable}
}
